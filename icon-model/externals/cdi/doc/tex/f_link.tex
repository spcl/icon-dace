There are two different interfaces for using {\CDI} functions in Fortran:
\texttt{cfortran.h} and the instrinsic \texttt{iso\_c\_binding} module from Fortran
2003 standard. At  first, the preparations for compilers without F2003
capabilities are described.\\\\
Every FORTRAN file that references {\CDI} functions or constants must contain
an appropriate \texttt{INCLUDE} statement before the first such reference:

\begin{verbatim}
   INCLUDE "cdi.inc"
\end{verbatim}

Unless the \texttt{cdi.inc} file is installed in a standard directory where
FORTRAN compiler always looks, you must use the \texttt{-I} option when invoking
the compiler, to specify a directory where \texttt{cdi.inc} is installed, for example:

\begin{verbatim}
   f77 -c -I/usr/local/cdi/include myprogram.f
\end{verbatim}

Alternatively, you could specify an absolute path name in the \texttt{INCLUDE}
statement, but then your program would not compile on another platform
where {\CDI} is installed in a different location.

Unless the {\CDI} library is installed in a standard directory where the linker
always looks, you must use the \texttt{-L} and \texttt{-l} options to links an object file that
uses the {\CDI} library. For example:

\begin{verbatim}
   f77 -o myprogram myprogram.o -L/usr/local/cdi/lib -lcdi
\end{verbatim}

Alternatively, you could specify an absolute path name for the library:

\begin{verbatim}
   f77 -o myprogram myprogram.o -L/usr/local/cdi/lib/libcdi
\end{verbatim}

If the {\CDI} library is using other external libraries, you must add this
libraries in the same way.
For example with the NetCDF library:

\begin{verbatim}
   f77 -o myprogram myprogram.o -L/usr/local/cdi/lib -lcdi \
                                -L/usr/local/netcdf/lib -lnetcdf
\end{verbatim}

For using the \texttt{iso\_c\_bindings} two things are necessary in a program or module

\begin{verbatim}
   USE ISO_C_BINDING
   USE mo_cdi
\end{verbatim}

The \texttt{iso\_c\_binding} module is included in \texttt{mo\_cdi}, but without
\texttt{cfortran.h} characters and character variables have to be handled separately.
Examples are available in section \ref{examples_f2003}.

After installation \texttt{mo\_cdi.o} and \texttt{mo\_cdi.mod} are located in the
library and header directory respectively. \texttt{cdilib.o} has to be
mentioned directly on the command line. It can be found in the
library directory, too. Depending on the {\CDI} configuration, a compile command
should look like this:

\begin{verbatim}
nagf95 -f2003 -g cdi_read_f2003.f90 -L/usr/lib -lnetcdf -o cdi_read_example 
                                    -I/usr/local/include 
                                    /usr/local/lib/cdilib.o /usr/local/lib/mo_cdi.o
\end{verbatim}


