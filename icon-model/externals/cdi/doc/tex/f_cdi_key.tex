

\subsection{Define a string from a key: \texttt{cdiDefKeyString}}
\index{cdiDefKeyString}
\label{cdiDefKeyString}

The function {\texttt{cdiDefKeyString}} defines a text string from a key.

\subsubsection*{Usage}

\begin{verbatim}
    INTEGER FUNCTION cdiDefKeyString(INTEGER cdiID, INTEGER varID, INTEGER key, 
                                     CHARACTER*(*) string)
\end{verbatim}

\hspace*{4mm}\begin{minipage}[]{15cm}
\begin{deflist}{\texttt{string}\ }
\item[\texttt{cdiID}]
CDI object ID (vlistID, gridID, zaxisID).
\item[\texttt{varID}]
Variable identifier or CDI\_GLOBAL.
\item[\texttt{key}]
The key to be searched.
\item[\texttt{string}]
The address of a string where the data will be read.

\end{deflist}
\end{minipage}

\subsubsection*{Result}

{\texttt{cdiDefKeyString}} returns CDI\_NOERR if OK.


\subsubsection*{Example}

Here is an example using {\texttt{cdiDefKeyString}} to define the name of a variable:

\begin{lstlisting}[language=Fortran, backgroundcolor=\color{pyellow}, basicstyle=\small, columns=flexible]

    INCLUDE 'cdi.inc'
       ...
    INTEGER vlistID, varID, status
       ...
    vlistID = vlistCreate()
    varID = vlistDefVar(vlistID, gridID, zaxisID, TIME_VARYING)
       ...
    status = cdiDefKeyString(vlistID, varID, CDI_KEY_NAME, "temperature")
       ...
\end{lstlisting}


\subsection{Get a string from a key: \texttt{cdiInqKeyString}}
\index{cdiInqKeyString}
\label{cdiInqKeyString}

The function {\texttt{cdiInqKeyString}} gets a text string from a key.

\subsubsection*{Usage}

\begin{verbatim}
    INTEGER FUNCTION cdiInqKeyString(INTEGER cdiID, INTEGER varID, INTEGER key, 
                                     CHARACTER*(*) string, INTEGER length)
\end{verbatim}

\hspace*{4mm}\begin{minipage}[]{15cm}
\begin{deflist}{\texttt{string}\ }
\item[\texttt{cdiID}]
CDI object ID (vlistID, gridID, zaxisID).
\item[\texttt{varID}]
Variable identifier or CDI\_GLOBAL.
\item[\texttt{key}]
The key to be searched.
\item[\texttt{string}]
The address of a string where the data will be retrieved.
                    The caller must allocate space for the returned string.
\item[\texttt{length}]
The allocated length of the string on input.
\end{deflist}
\end{minipage}

\subsubsection*{Result}

{\texttt{cdiInqKeyString}} returns CDI\_NOERR if key is available.


\subsubsection*{Example}

Here is an example using {\texttt{cdiInqKeyString}} to get the name of the first variable:

\begin{lstlisting}[language=Fortran, backgroundcolor=\color{pyellow}, basicstyle=\small, columns=flexible]

    INCLUDE 'cdi.inc'
       ...
    #define STRLEN 256
       ...
    INTEGER streamID, vlistID, varID, status
    INTEGER length = STRLEN
    CHARACTER name[STRLEN]
       ...
    streamID = streamOpenRead(...)
    vlistID = streamInqVlist(streamID)
       ...
    varID = 0
    status = cdiInqKeyString(vlistID, varID, CDI_KEY_NAME, name, length)
       ...
\end{lstlisting}


\subsection{Define an integer value from a key: \texttt{cdiDefKeyInt}}
\index{cdiDefKeyInt}
\label{cdiDefKeyInt}

The function {\texttt{cdiDefKeyInt}} defines an integer value from a key.

\subsubsection*{Usage}

\begin{verbatim}
    INTEGER FUNCTION cdiDefKeyInt(INTEGER cdiID, INTEGER varID, INTEGER key, 
                                  INTEGER value)
\end{verbatim}

\hspace*{4mm}\begin{minipage}[]{15cm}
\begin{deflist}{\texttt{cdiID}\ }
\item[\texttt{cdiID}]
CDI object ID (vlistID, gridID, zaxisID).
\item[\texttt{varID}]
Variable identifier or CDI\_GLOBAL.
\item[\texttt{key}]
The key to be searched.
\item[\texttt{value}]
An integer where the data will be read.

\end{deflist}
\end{minipage}

\subsubsection*{Result}

{\texttt{cdiDefKeyInt}} returns CDI\_NOERR if OK.



\subsection{Get an integer value from a key: \texttt{cdiInqKeyInt}}
\index{cdiInqKeyInt}
\label{cdiInqKeyInt}

The function {\texttt{cdiInqKeyInt}} gets an integer value from a key.

\subsubsection*{Usage}

\begin{verbatim}
    INTEGER FUNCTION cdiInqKeyInt(INTEGER cdiID, INTEGER varID, INTEGER key, 
                                  INTEGER value)
\end{verbatim}

\hspace*{4mm}\begin{minipage}[]{15cm}
\begin{deflist}{\texttt{cdiID}\ }
\item[\texttt{cdiID}]
CDI object ID (vlistID, gridID, zaxisID).
\item[\texttt{varID}]
Variable identifier or CDI\_GLOBAL.
\item[\texttt{key}]
The key to be searched..
\item[\texttt{value}]
The address of an integer where the data will be retrieved.

\end{deflist}
\end{minipage}

\subsubsection*{Result}

{\texttt{cdiInqKeyInt}} returns CDI\_NOERR if key is available.



\subsection{Define a floating point value from a key: \texttt{cdiDefKeyFloat}}
\index{cdiDefKeyFloat}
\label{cdiDefKeyFloat}

The function {\texttt{cdiDefKeyFloat}} defines a {\CDI} floating point value from a key.

\subsubsection*{Usage}

\begin{verbatim}
    INTEGER FUNCTION cdiDefKeyFloat(INTEGER cdiID, INTEGER varID, INTEGER key, 
                                    REAL*8 value)
\end{verbatim}

\hspace*{4mm}\begin{minipage}[]{15cm}
\begin{deflist}{\texttt{cdiID}\ }
\item[\texttt{cdiID}]
CDI object ID (vlistID, gridID, zaxisID).
\item[\texttt{varID}]
Variable identifier or CDI\_GLOBAL.
\item[\texttt{key}]
The key to be searched
\item[\texttt{value}]
A double where the data will be read

\end{deflist}
\end{minipage}

\subsubsection*{Result}

{\texttt{cdiDefKeyFloat}} returns CDI\_NOERR if OK.



\subsection{Get a floating point value from a key: \texttt{cdiInqKeyFloat}}
\index{cdiInqKeyFloat}
\label{cdiInqKeyFloat}

The function {\texttt{cdiInqKeyFloat}} gets a floating point value from a key.

\subsubsection*{Usage}

\begin{verbatim}
    INTEGER FUNCTION cdiInqKeyFloat(INTEGER cdiID, INTEGER varID, INTEGER key, 
                                    REAL*8 value)
\end{verbatim}

\hspace*{4mm}\begin{minipage}[]{15cm}
\begin{deflist}{\texttt{cdiID}\ }
\item[\texttt{cdiID}]
CDI object ID (vlistID, gridID, zaxisID).
\item[\texttt{varID}]
Variable identifier or CDI\_GLOBAL.
\item[\texttt{key}]
The key to be searched.
\item[\texttt{value}]
The address of a double where the data will be retrieved.

\end{deflist}
\end{minipage}

\subsubsection*{Result}

{\texttt{cdiInqKeyFloat}} returns CDI\_NOERR if key is available.



\subsection{Define a byte array from a key: \texttt{cdiDefKeyBytes}}
\index{cdiDefKeyBytes}
\label{cdiDefKeyBytes}

The function {\texttt{cdiDefKeyBytes}} defines a byte array from a key.

\subsubsection*{Usage}

\begin{verbatim}
    INTEGER FUNCTION cdiDefKeyBytes(INTEGER cdiID, INTEGER varID, INTEGER key, 
                                    unsigned CHARACTER*(*) bytes, INTEGER length)
\end{verbatim}

\hspace*{4mm}\begin{minipage}[]{15cm}
\begin{deflist}{\texttt{length}\ }
\item[\texttt{cdiID}]
CDI object ID (vlistID, gridID, zaxisID).
\item[\texttt{varID}]
Variable identifier or CDI\_GLOBAL.
\item[\texttt{key}]
The key to be searched.
\item[\texttt{bytes}]
The address of a byte array where the data will be read.
\item[\texttt{length}]
Length of the byte array

\end{deflist}
\end{minipage}

\subsubsection*{Result}

{\texttt{cdiDefKeyBytes}} returns CDI\_NOERR if OK.



\subsection{Get a byte array from a key: \texttt{cdiInqKeyBytes}}
\index{cdiInqKeyBytes}
\label{cdiInqKeyBytes}

The function {\texttt{cdiInqKeyBytes}} gets a byte array from a key.

\subsubsection*{Usage}

\begin{verbatim}
    INTEGER FUNCTION cdiInqKeyBytes(INTEGER cdiID, INTEGER varID, INTEGER key, 
                                    unsigned CHARACTER*(*) bytes, INTEGER length)
\end{verbatim}

\hspace*{4mm}\begin{minipage}[]{15cm}
\begin{deflist}{\texttt{length}\ }
\item[\texttt{cdiID}]
CDI object ID (vlistID, gridID, zaxisID).
\item[\texttt{varID}]
Variable identifier or CDI\_GLOBAL.
\item[\texttt{key}]
The key to be searched.
\item[\texttt{bytes}]
The address of a byte array where the data will be retrieved.
                    The caller must allocate space for the returned byte array.
\item[\texttt{length}]
The allocated length of the byte array on input.
\end{deflist}
\end{minipage}

\subsubsection*{Result}

{\texttt{cdiInqKeyBytes}} returns CDI\_NOERR if key is available.

