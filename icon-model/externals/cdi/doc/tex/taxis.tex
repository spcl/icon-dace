This section contains functions to define a new Time axis
and to get information from an existing T-axis.
A T-axis  object is necessary to define the time axis of a dataset
and must be assiged to a variable list using \htmlref{\texttt{vlistDefTaxis}}{vlistDefTaxis}.
The following different Time axis types are available:

\vspace*{3mm}
\hspace*{8mm}\begin{minipage}{15cm}
\begin{deflist}{\large\texttt{TAXIS\_RELATIVE \ \ }}
\item[\large\texttt{TAXIS\_ABSOLUTE}]  Absolute time axis    
\item[\large\texttt{TAXIS\_RELATIVE}]  Relative time axis
\end{deflist}
\end{minipage}
\vspace*{3mm}

An absolute time axis has the current time to each time step.
It can be used without knowledge of the calendar.

A relative time is the time relative to a fixed reference time.
The current time results from the reference time and the elapsed interval.
The result depends on the used calendar.
CDI supports the following calendar types:

\vspace*{3mm}
\hspace*{8mm}\begin{minipage}{15cm}
\begin{deflist}{\large\texttt{CALENDAR\_PROLEPTIC \ \ }}
\item[\large\texttt{CALENDAR\_STANDARD}]  Mixed Gregorian/Julian calendar.
\item[\large\texttt{CALENDAR\_PROLEPTIC}]  Proleptic Gregorian calendar. This is the default.
\item[\large\texttt{CALENDAR\_360DAYS }]  All years are 360 days divided into 30 day months.
\item[\large\texttt{CALENDAR\_365DAYS }]  Gregorian calendar without leap years, i.e., all years are 365 days long.
\item[\large\texttt{CALENDAR\_366DAYS }]  Gregorian calendar with every year being a leap year, i.e., all years are 366 days long.
\end{deflist}
\end{minipage}
